% --------------------------------------------------------------------------------------------------------------
%
% Copyright 2020-2022 Robert Bosch GmbH

% Licensed under the Apache License, Version 2.0 (the "License");
% you may not use this file except in compliance with the License.
% You may obtain a copy of the License at

% http://www.apache.org/licenses/LICENSE-2.0

% Unless required by applicable law or agreed to in writing, software
% distributed under the License is distributed on an "AS IS" BASIS,
% WITHOUT WARRANTIES OR CONDITIONS OF ANY KIND, either express or implied.
% See the License for the specific language governing permissions and
% limitations under the License.
%
% --------------------------------------------------------------------------------------------------------------

The JsonPreprocessor is a Python3 package which provides additional features for
JSON files.

The sources of the JsonPreprocessor are available in \href{https://github.com/test-fullautomation/python-jsonpreprocessor}{GitHub}.

These are:
\begin{enumerate}
\item \texttt{import}\\
   ability to (nested) import other JSON files. Users can create separate JSON files and then import 
   them to another JSON file.
\item \texttt{\$\{parameter\}-syntax}\\
   allows users to use already defined parameters by means of \pcode{\$\{parameter\}}-syntax in JSON files.
\item \texttt{overwrite}\\
   overwrite already existing parameters with new values from later loaded JSON files.
\item \texttt{Python-like syntax}\\
   accept Python like ``True``, ``False`` and ``None`` in JSON files. 
\item \texttt{comments}\\
  provide the possibility to comment out parts of the JSON file content for documentation purposes.
\end{enumerate}

The JsonPreprocessor returns as result a dictionary object of the deserialized 
preprocessed JSON data.


