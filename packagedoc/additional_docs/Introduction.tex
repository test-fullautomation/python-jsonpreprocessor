% Copyright 2020-2024 Robert Bosch GmbH

% Licensed under the Apache License, Version 2.0 (the "License");
% you may not use this file except in compliance with the License.
% You may obtain a copy of the License at

% http://www.apache.org/licenses/LICENSE-2.0

% Unless required by applicable law or agreed to in writing, software
% distributed under the License is distributed on an "AS IS" BASIS,
% WITHOUT WARRANTIES OR CONDITIONS OF ANY KIND, either express or implied.
% See the License for the specific language governing permissions and
% limitations under the License.

\textbf{J}ava\textbf{S}cript \textbf{O}bject \textbf{N}otation (\textbf{JSON}) is a text-based format for storing any user defined data and can also be used
for data interchange between different applications.

But this format has some limitations and the \pkg\ has been introduced to fill the gaps.

The \pkg\ extends the JSON format by the following features:

\begin{enumerate}
   \item Parts of a JSON file can be commented out
   \item A JSON file can import other JSON files (nested imports)
   \item Parameter can be defined, referenced and overwritten (follow up definitions in configuration files overwrite previous definitions of the same parameter)
   \item Also Python specific keywords like \pcode{True}, \pcode{False} and \pcode{None} can be used (additionally to the corresponding JSON keywords \pcode{true},
         \pcode{false} and \pcode{null})
\end{enumerate}

The main goal of the \pkg\ is to support huge sets of parameters for complex projects. And the features of the \pkg\ support this complexity:

\begin{enumerate}
   \item Like in usual programming languages code comments are useful to explain the meaning of the defined parameters.
   \item Splitting all required parameters into several JSON files - that can import each other - enables to distinguish e.g. between local and global parameters
         or between specific and common parameters. Another advantage of a file split is: Smaller files with a more specific content are easier to maintain than
         a huge single file that contains all.
   \item A possible use case for a file split would be to have a software containing several different components with each component requires an individual set of
         parameters - and therefore an own JSON file. Additionally all components also require a common set of parameters. In this case all common parameters can be defined
         within an own JSON file that is imported into all other JSON files containing the specific values. This procedure avoids redundancy in parameter definitions.
   \item Parameters can be initialized in common JSON files and overwritten in specific JSON files that import the common ones.
\end{enumerate}

But this has consequences: The new features cause some deviations from JSON standard. 

These deviations harm the syntax highlighting of editors and also cause invalid findings of JSON format related static code checkers.

To avoid conflicts between the standard JSON format and the extended JSON format described here, the \pkg\ uses the alternative file extension
\pcode{.jsonp} for all JSON files.

\newpage

\textbf{References:}

The \pkg\ is hosted in PyPi (recommended for users) and in GitHub (recommended for developers):

\begin{itemize}
   \item \href{https://pypi.org/project/JsonPreprocessor/}{JsonPreprocessor in PyPi}
   \item \href{https://github.com/test-fullautomation/python-jsonpreprocessor}{JsonPreprocessor in GitHub}
\end{itemize}

Details about how to get the \pkg\ can be found in the
\href{https://github.com/test-fullautomation/python-jsonpreprocessor/blob/develop/README.rst}{README}.

For the development environment \textbf{VSCodium} an extension is available to support the extended JSON format of the \pkg:
\href{https://github.com/test-fullautomation/vscode-jsonp}{vscode-jsonp}

